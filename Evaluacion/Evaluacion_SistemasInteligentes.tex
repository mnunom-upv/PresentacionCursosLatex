%------------------------------------------------
\section{Evaluación}
%------------------------------------------------

\begin{frame}
\frametitle{Evaluación (1)}
\begin{itemize}
\item Para cada unidad del curso, se consideran 3 aspectos:
\begin{itemize}
\item Ejercicios o investigaciones especiales (1)- 25\%
\item Proyecto Individual  - 35\%
\item Proyecto en Equipo - 40\%
\end{itemize}
\item Para aprobar el curso, es obligatorio:
\begin{itemize} 
\item Tener calificacion aprobatoria en todas las unidades (100-100-40 no da calificación aprobatoria).
\item Tener por lo menos dos asesorías por semana (Registrarlas por semana, no 30 asesorías al final del cuatrimestre)
\item Cumplir con el 80\% de asistencia mínimo, incluyendo aquellas inasistencias justificadas debidamente
\end{itemize}
\end{itemize}
\end{frame}

\begin{frame}
\frametitle{Evaluación (2)}
Para cada unidad, habra sesiones de ``teoria'', sesiones de seguimiento de proyectos y sesiones de esparci  miento
\begin{itemize}
\item En las sesiones de teoria, el profesor presentara uno o varios temas
\item En las sesiones de seguimiento de proyectos, de manera aleatoria se nombrara al integrante de equipo individual o en equipo. En el caso de que un integrante individual no responda, se le bajarán 5 puntos a su calificación del proyecto
\item En las sesiones de esparcimiento, se permitirá a los estudiantes trabajar en proyectos pendientes, pero se contabilizará la asistencia. 
\end{itemize}
\end{frame}

\begin{frame}
\frametitle{Evaluación (3)}
Sesiones de Seguimento de proyectos
\begin{itemize}
\item En el caso de que el integrante del equipo seleccionado aleatoriamente no responda satisfactoriamente lo cuestionado, se le bajaran 5 puntos a su calificacion del proyecto a todos los integrantes del equipo
\item En el caso de los proyectos en equipo, el integrante seleccionado es aleatorio. Si en una primera ronda le toco al integrante A, en una segunda ronda posiblemente le toque al integrante B
\end{itemize}
\end{frame}



\begin{frame}
\frametitle{Evaluación (4)}
Lo que se debe presentar en una sesion de seguimiento de proyectos
\begin{itemize}
\item En un trabajo individual
\begin{itemize}
\item Compartir pantalla de la ejecucion del avance del proyecto
\item Explicar con recursos multimedia los pasos para la resolucion del proyecto
\item Establecer el avance desde la ultima entrega
\end{itemize}
\item En un trabajo grupal
\begin{itemize}
\item Compartir pantalla de la ejecucion del avance del proyecto
\item Explicar con recursos multimedia los pasos para la resolucion del proyecto
\item Desglosar como se repartio el trabajo entre los integrantes del equipo
\item Establecer el avance desde la ultima entrega
\end{itemize}
\end{itemize}
\end{frame}



\begin{frame}
\frametitle{Evaluación (5)}
Acerca de los proyectos
\begin{itemize}
\item Aleatorios y DIFERENTES para la mayoria (preferentemente para cada integrante)
\item Equipos: Proyectos diferentes para cada equipo, e Integrantes de los mismos formados de manera ALEATORIA!!
\end{itemize}
\end{frame}



\begin{frame}
\frametitle{Fragmentaci\'on de equipos}
\begin{itemize}
\item Si llegar a ocurrir que en un proyecto en equipo no hay un acuerdo para trabajar en equipo (Hay dos o mas entregas del proyecto asignado por partes diferentes dentro del mismo equipo)
\begin{block}{Penalizaci\'on}
Cada ``fragmento'' de equipo recibe una penalizacion de 25 puntos mas las penalizaciones acumuladas por otros rubros. 
\end{block}
\item esta regla \textbf{NO APLICA} cuando hay uno o varios ``desertores'' del equipo (y hay una sola entrega del proyectos en equipo)
\end{itemize}

\end{frame}



\begin{frame}
\frametitle{Acerca de Excensión}
\begin{itemize}
\item Cúando el profesor realizar alguna mecánica para excentar un proyecto (Individual/Equipo/Asignación especial) y uno o varios estudiantes completan lo solicitado, existen dos posibilidades:
\begin{itemize}
\item El estudiante acepta excentar la elaboración de dicho proyecto o actividad, pero al hacer esto asume que la calificación asignada es 70.
\item El estudiante decide hacer el proyecto a pesar de haber excentado. En este caso el estudiante se hace acreedor a 20 puntos que puede aplicar sobre la calificación de dicho proyecto. 

\end{itemize}
\end{itemize}


\end{frame}



%\begin{frame}
%\frametitle{Evaluación (6)}
%Acerca de la participacion
%\begin{itemize}
%\item Se selecciona al azar un estudiante, existen varias posibilidades:
%\begin{itemize}
%\item Esta presente, puede abrir camara, microfono y compartir escritorio para responder lo solicitado, y responde   $\rightarrow$  POSITIVA
%\item Esta presente, NO puede abrir camara, ni microfono, y no comparte escritorio lo solicitado, no es necesario que responda  $\rightarrow$ NEGATIVA
%\item Esta presente, pero decide no participar $\rightarrow$ NEGATIVA
%\item NO esta presente $\rightarrow$ NEGATIVA
%\end{itemize}
%\item La calificacion de Participación es proporcional al porcentaje de participaciones positivas (7/10, 3/5, 0/4, 3/3). 
%\end{itemize}
%\end{frame}

