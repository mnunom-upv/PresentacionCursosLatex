%------------------------------------------------
\section{Evaluación}
%------------------------------------------------

\begin{frame}
\frametitle{Evaluación (1)}
Para cada unidad del curso, se consideran los siguientes aspectos:
\begin{itemize}
\item Exposicion de Tema Selecto 1- 10\%
\item Exposicion de Tema Selecto 2- 10\%
\item Proyecto Individual 1 - 20\%
\item Proyecto Individual 2 - 20\%
\item Proyecto Individual 3 - 20\%
\item Proyecto Individual 4 - 20\%
\end{itemize}
Para aprobar el curso, es obligatorio entregar todos los proyectos solicitados
%Para tener derecho a una evaluacion de recuperacion de la unidad, el estudiante debe haber cumplido con 2 /3 requisitos de la unidad, y esta recuperación aplica solamente UNA VEZ a lo largo del cuatrimestre.
\end{frame}

\begin{frame}
\frametitle{Evaluación (2)}
Para cada unidad, habra sesiones de ``teoria'', sesiones de seguimiento de proyectos y sesiones de esparcimiento
\begin{itemize}
\item En las sesiones de teoria, el profesor presentara uno o varios temas
\item En las sesiones de seguimiento de proyectos, de manera aleatoria se nombrara al integrante de equipo individual o en equipo. En el caso de que un integrante individual no responda, se le bajarán 5 puntos a su calificación del proyecto
\item En las sesiones de esparcimiento, se permitirá a los estudiantes trabajar en proyectos pendientes, pero se contabilizará la asistencia. 
\end{itemize}
\end{frame}

\begin{frame}
\frametitle{Evaluación (3)}
Sesiones de Seguimento de proyectos
\begin{itemize}
\item Programadas generalmente una semana despues de haber sido asignado el proyecto
\item Los estudiantes deben exponer el avance del proyecto, as\'i como los retos y dificultades sorteados
%\item En el caso de que el integrante del equipo seleccionado aleatoriamente no responda satisfactoriamente lo cuestionado, se le bajaran 5 puntos a su calificacion del proyecto a todos los integrantes del equipo
%\item En el caso de los proyectos en equipo, el integrante seleccionado es aleatorio. Si en una primera ronda le toco al integrante A, en una segunda ronda posiblemente le toque al integrante B
\end{itemize}
\end{frame}


\begin{frame}
\frametitle{Evaluación (4)}
Acerca de los proyectos
\begin{itemize}
\item Proyectos diferentes para los integrantes de la clase 
\item Asignados de manera aleatoria para cada integrante de la clase
\end{itemize}
\end{frame}


