

%------------------------------------------------
\section{Entregables}
%------------------------------------------------



\begin{frame}
\frametitle{Reporte Técnico de Desarrollo de Práctica}
\begin{itemize}
\item Para cada práctica realizada, entregar un documento (\textbf{únicamente en formato PDF*}) con las siguientes secciones:
\begin{itemize}
\item Introducción
\item Desarrollo Experimental
\item Resultados
\item Conclusiones
\item \textbf{Referencias}
\end{itemize}
\item Para GENERAR este reporte es necesario utilizar la plantilla en LATEX (\textbf{únicamente usando LATEX*}) localizada en el siguiente enlace:
\url{https://www.overleaf.com/read/dgkhvfwnygvc}
\end{itemize}
\end{frame}

\begin{frame}
\frametitle{Reporte Técnico de Desarrollo de Práctica}
\begin{itemize}
\item Bajo ninguna circunstancia deben incluir \textbf{CÓDIGO FUENTE}. Si pueden incluir diagrama de flujo, Pseudocódigo, Diagrama E-R, Diagrama de Clases, de Casos de USO, etc. De incluir código fuente, solo generá una penalizaci\'on a la calificación del proyecto.  
\item En caso de trabajos indivudales o en EQUIPO, deben emplear la plantilla LaTex que se provee. En caso de utilizar algo diferente a LaTex u otra plantilla de LaTex, la calificación proporcional del informe será \textbf{DESESTIMADA}. 
%\item En caso de trabajos en equipo, se debe agregar los integrantes al inicio del INFORME. \textbf{El trabajo solo cuenta para aquellos integrantes mencionados en el informe (y que dicho nombre se encuentre registrado tal cual en la lista). Una vez ENTREGADO, si hay OMISIONES de los integrantes, no se realizará CORRECCION alguna, se debe asumir la consecuencias que esto conlleva. }
\end{itemize}

\end{frame}


\begin{frame}
\frametitle{Ponderación del Informe en la Calificación del Proyecto}
\begin{itemize}
\item Proyecto: 66 Puntos
\begin{itemize}
\item Ejecución y Funcionalidad: 45 Puntos
\item Modularidad: 13 Puntos
\item Documentación: 8 Puntos
\end{itemize}
\item Informe: 34 Puntos
\begin{itemize}
\item Uso adecuado de Latex: 5 Puntos
\item Organizaci\'on y Redacci\'on: 6 Puntos
\item Referencias en formato adecuado: 8 Puntos
\item Evidencia del trabajo realizado: 8 Puntos
\item Sin faltas de ortografía ni errores de dedo: 7 Puntos
\end{itemize}
\end{itemize}
\end{frame}







%------------------------------------------------



\begin{frame}
\frametitle{Entregables de proyecto individual}
En cada entrega, subir un archivo .ZIP, cuyo nombre de archivo debe seguir la siguiente especificación (todo en minúsculas , sin ESPACIOS):
\begin{itemize}
\item \textbf{Clave de GRUPO (incluir guión medio)}
\item \textbf{Nombre del integrante iniciando por apellido paterno, SIN ESPACIOS y separado por guion bajo}
\item \textit{clavegrupo}\_nuno\_maganda\_marco\_aurelio.zip,  \textit{clavegrupo}\_nuno\_maganda\_marco\_aurelio.pdf, etc. 
\end{itemize}
%\textbf{Clave de GRUPO seguido del nombre del integrante iniciando por apellido paterno, SIN ESPACIOS y separado por guion bajo}. 
\begin{itemize}
\item El contenido de dicho archivo debe ser el siguiente:
\begin{enumerate}
\item Archivo .ZIP con el código fuente.
\item Archivo instalador de la aplicación (.APK).  
\item Archivo PDF con el informe. 
\end{enumerate}
\item Mismo formato de NOMBRE de archivo del ENTREGABLE principal para el nombre de los archivos al interior del ZIP
\end{itemize}

\end{frame}


\begin{frame}
\frametitle{Entregables de proyecto individual - Ejemplos}
\begin{itemize}
\item Este es un ejemplo, pero nunca falta el listo que lo entrega tal cual!
\item \textbf{Archivo Principal:} iti-271086\_nuno\_maganda\_marco\_aurelio.zip
\item Contenido de dicho archivo:

\begin{itemize}
\item \textbf{iti-271086\_nuno\_maganda\_marco\_aurelio.zip} (Codigo fuente)
\item \textbf{iti-271086\_nuno\_maganda\_marco\_aurelio.apk} (Instalable)
\item \textbf{iti-271086\_nuno\_maganda\_marco\_aurelio.pdf} (Informe)
\end{itemize}

\end{itemize}

\textbf{Cuatro cuatrimestres ignorando estas instrucciones, ya deber\'ia poner nombres y apellidos. El que lo haga ser\'an sentadillas o p\'aginas, ustedes escojan!}

\end{frame}



%\begin{frame}
%En cada entrega, \textbf{UN SOLO INTEGRANTE DEL EQUIPO} deberá subir un archivo .ZIP, cuyo nombre de archivo debe seguir la siguiente especificación (todo en minúsculas):
%\frametitle{Entregables de proyecto en equipo}
%\begin{itemize}
%\item \textbf{Clave de GRUPO (incluir guión)}
%\item \textbf{Palabra equipo seguido del numero de equipo (usando dos digitos)}
%\end{itemize}

%\begin{enumerate}
%\item Reporte PDF.
%\item Código fuente en una carpeta.
%\end{enumerate}
%Ejemplos: \textit{claveGrupo}\_equipo\_01.zip, iti-27798\_equipo\_01.pdf, etc
%\end{frame}

\begin{frame}
\frametitle{Nombres de Archivos Entregables}
En el caso de nombres y apellidos acentuados, con dieresis o con virgulilla (\textasciitilde{}), sustituir de acuerdo con las siguientes reglas:
\begin{itemize}
\item Sustituir N/n por \~N/\~n
\item Sustituir A/a por \'A/\'a
\item Sustituir E/e por \'E/\'e
\item Sustituir I/i por \'I/\'i
\item Sustituir O/o por \'O/\'o
\item Sustituir U/u por \'U/\'u
\item Sustituir U/u por \"U/\"u
\end{itemize}
\end{frame}



\begin{frame}
\frametitle{Codigo Fuente del Proyecto (Solo aplica para entrega de APP m\'oviles)}
\begin{itemize}
\item Se debe extraer utilizando el SCRIPT para dicho prop\'osito (SOLO EXISTE HAY PARA LINUX, usuarios Windows deben programar el SUYO o replicar los pasos del SCRIPT Linux para sus entrega) que se proveerá en el CLASSROOM para extraer los archivos necesarios
\item \textbf{De ninguna manera se debe compactar la carpeta completa del PROYECTO. De hacer esto, habr\'a una penalizaci\'on de 20 PUNTOS}
\end{itemize}

\end{frame}
