\begin{frame}
\frametitle{Sistema Operativo Oficial}

\textbf{LINUX}
\begin{block}{Recomendaciones}
\begin{itemize}
\item No es obligatorio instalarlo, pero es recomendable por cuestión de desempeño.
\item Si no quieren formatear computadora, se recomienda utilizar un HD booteable (SSD con persistencia) y bootear desde su laptop o computadora.
\item Si lo instalan de manera nativa, puede ser cualquier distribución \textbf{(Mint, Ubuntu, Lubuntu, Xubuntu, Debian)}.
\end{itemize}
\end{block}
%\textbf{** Tienen la opción de no INSTALAR LINUX, pero la evaluación será realiza en una PC con Linux instalado}
\begin{block}{Dispositivo Físico con Android}
\begin{itemize}
\item Teléfono Inteligente/Tablet con Android Instalado (No afecta si no es la última versión)
\end{itemize}
\end{block}
\end{frame}


\begin{frame}
\frametitle{Software Utilizado}
Sobre una instalación de Linux, se debe instalar lo siguiente:
\begin{itemize}
\item Android Studio
\item Scrcpy (\url{https://github.com/Genymobile/scrcpy})
\item Navegador Chrome/Firefox actualizado
\item LaTeX para edición de reportes
\end{itemize}
\end{frame}

\begin{frame}
\frametitle{Lenguajes Utilizados}
Los lenguajes soportados por Android Studio son:
\begin{itemize}
\item Kotlin
\item Java
\end{itemize}
Se hará enfásis en el lenguaje Kotlin, dado que es el lenguaje recomendado para nuevas aplicaciones, sin embargo, podría trabajarse también en Java dependiendo del proyecto
\end{frame}

