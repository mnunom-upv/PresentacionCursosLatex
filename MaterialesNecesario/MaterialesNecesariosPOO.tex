\begin{frame}
\frametitle{Sistema Operativo Oficial}

\textbf{LINUX}
\begin{block}{En orden de dificultad}
\begin{itemize}
\item Linux instalado de manera emulada usando VirtualBox o VMWare.
\item Crear una USB o HD booteable (con persistencia) y bootear desde su laptop solo para las clases y los proyectos.
\item Linux instalado de manera nativa. Distribuciones recomendadas: \textbf{Mint, Ubuntu, Lubuntu, Xubuntu, Debian}
\end{itemize}
\end{block}
\textbf{** Tienen la opción de no INSTALAR LINUX, pero la evaluación será realiza en una PC con Linux instalado}
\end{frame}


\begin{frame}
\frametitle{Software Utilizado}
Sobre una instalación de Linux, se debe instalar lo siguiente:
\begin{itemize}
\item Navegador Chrome/Firefox actualizado
\item LaTeX para edición de reportes (\textit{sudo apt-get install texlive-full}).
\item C++, Java y Python3.
\item Qt5 (para C++) y PyQt5 (para Python3).
\item Otras librerias (se espeficarán conforme se vayan utilizando).
\end{itemize}
\end{frame}

